\chapter{序論}
\section{研究背景と研究目的} % (fold)
\label{sec:研究背景と研究目的}


多様な新しい種類のデータが大量に生成されるビッグデータの時代において,データを人手によらないで分類/認識する汎用的なパターン認識手法の重要性が高まっている.
しかし,一般的な統計的パターン認識手法では,パラメータ選択を含めモデルを人手で適切に設計する手間が大きく,新種のデータを分析するのは容易ではない.
これに対して,圧縮ベースパターン認識は,データ圧縮アルゴリズムを用いたパラメータフリーなパターン認識手法であり,1次元に表現されたデータであれば何でも解析可能な汎用手法であるため,近年盛んに研究されている.

圧縮ベースパターン認識は(1)圧縮後のファイルサイズを特徴として利用する手法と(2)圧縮時に生成される辞書を元データの要約として用いてデータ間の距離を測る手法に大別される.
前者の例としては,2つのデータ$x,y$を結合したファイルを圧縮しそのサイズから,$x$と$y$の距離を測る Normalized Compression Distance (NCD) \cite{NCD} が
広く知られている.
また,与えられたデータを複数の辞書で圧縮し,それぞれの辞書で求められた圧縮率を並べて特徴ベクトルとするPattern Representation on Data Compression(PRDC) \cite{PRDC} もよく用いられる.
後者の例としては,LZW圧縮で生成される辞書をデータの要約とみなし,辞書間で類似度を計算する Normalized Dictionary Distance (NDD) \cite{NDD}や単語の多重度も考慮したNormalized Multiset Distance (NMD) \cite{NMD}が提案されている.

本研究は,圧縮ベースパターン認識に有用な特徴の探求を目的とする.具体的には,従来手法であるPRDCとNMDを取り上げ,PRDCに対しては圧縮後のファイルから,NMDに関しては圧縮辞書からそれぞれ新しい特徴を抽出することで,認識精度を向上できることを示す.

また,新しい特徴を抽出する際,特に考慮した点は以下の2つである.
\begin{itemize}
	\item 類似度計算におけるパラメータを増やさない
	\item 特徴抽出のために複雑な計算をしない
\end{itemize}
これにより,圧縮ベースパターン認識の特徴である手軽さを損なわず,実用的な手法の提案を目的とした.

% section 研究背景と研究目的 (end)

\section{本論文の構成} % (fold)
\label{sec:本論文の構成}
以下に本論文の構成を述べる.

\begin{description}
	\item{第2章} 本研究で用いる圧縮アルゴリズムであるLZWの説明と,圧縮ベースパターン認識の代表的な2種類の手法について述べる.
	\item{第3章} 圧縮後のファイルから特徴抽出を行う手法であるPRDCに着目し,圧縮後のファイルから別の特徴を新たに抽出することでPRDCの改良を行う.
	\item{第4章} 圧縮る辞書を元データの要約として用いてデータ間の距離を測る手法である NMD を取り上げ,圧縮辞書からの新しい特徴抽出による NMD の改良手法を説明する.
	\item{第5章} 本研究で提案した手法の評価実験を画像と時系列データに対して行い,その有効性を示す.
	\item{第6章} 本研究のまとめと今後の課題について述べる.
\end{description}
% section 本論文の構成 (end)